\documentclass{article}

% if you need to pass options to natbib, use, e.g.:
% \PassOptionsToPackage{numbers, compress}{natbib}
% before loading nips_2017
%
% to avoid loading the natbib package, add option nonatbib:
% \usepackage[nonatbib]{nips_2017}

\usepackage[final]{nips_2017}


\usepackage[utf8]{inputenc} % allow utf-8 input
\usepackage[T1]{fontenc}    % use 8-bit T1 fonts
\usepackage{hyperref}       % hyperlinks
\usepackage{url}            % simple URL typesetting
\usepackage{booktabs}       % professional-quality tables
\usepackage{amsfonts}       % blackboard math symbols
\usepackage{nicefrac}       % compact symbols for 1/2, etc.
\usepackage{microtype}      % microtypography

% Choose a title for your submission
\title{Natural Language Understanding - Project 2}


\author{Gabriel Hayat \qquad Hidde Lycklama \qquad Yiji He \qquad Arthur Deschamps}

\begin{document}
% \nipsfinalcopy is no longer used

\maketitle

% We do not requrire you to write an abstract. Still, if you feel like it, please do so.
%\begin{abstract}
%\end{abstract}

Feel free to add more sections but those listed here are strongly recommended.
\section{Introduction}
You can keep this short. Ideally you introduce the task already in a way that highlights the difficulties  your method will tackle.
\section{Methodology}
Your idea. You can rename this section if you like. Early on in this section -- but not necessarily first -- make clear what category your method falls into: Is it generative? Discriminative? Is there a particular additional data source you want to use?
\section{Model}
The math/architecture of your model. This should formally describe your idea from above. If you really want to, you can merge the two sections.
\section{Training}
What is your objective? How do you optimize it?

\section{Experiments}
This {\bf must} at least include the accuracy of your method on the validation set.
\section{Conclusion}
You can keep this short, too.
\end{document}
